% !TEX root = /home/computer/ucsc/master-2/quarter-2/advanced-fluids/master.tex
\lecture{1}{Tue 04 Jan 2022 13:52}{The Governing Equations of Fluid Dynamics}

\subsectionfont{\fontsize{10}{10}\selectfont}

\section{Mass Conservation}%
\textbf{Eulerian Description} 

\textbf{Lagrangian Description} 
\[
\frac{D \rho}{Dt} = - \rho \nabla\cdot u
.\] 

Method of characteristics help this make more sense
\section{Momentum Conservation}%
\begin{align}
  \rho \frac{D \vec{u}}{Dt} &= \sum \vec{F} \\
                            &= \underbrace{\nabla p}_\text{pressure}
                            + \underbrace{\nabla\cdot \Pi}_\text{viscosity} + F_{\text{external}}
\end{align}

where, for Newtonian fluid
\[
  \Pi = (k - \frac{2}{3}\mu)\nabla\cdot \vec{u}
.\] 
\section{Equation of State (EOS)}%

Comes from thermodynamics, and is a property of
the fluid. Relates various thermodynamics quantities to one another. e.g. $p,
\rho, T, \underbrace{s}_\text{specific entropy / unit mass},
\underbrace{e}_\text{eternal energy / unit mass}$. $e$ energy inside gas if on
the whole it is not moving?. Equation of state is needed to solve fluid
problems.

For a single-component fluid or gas (e.g. water, O2), an equation of state
relates one thermodynamic property to two others, eg
\begin{align*}
  p &= f( \rho, T)
  e = f(s, p)
\end{align*}

All of these e.o.s are equivalent. They can be derived from each other.


Examples:

\textbf{Perfect gas:} 
\[
p = R \rho T
.\] 

where $R = $ gas constant (specific to gas considered). Or  $R = \frac{Ru}{mg}$
where $Ru$ is the universal gas constant, and $mg$ is the molecular weight of
gas.

\textbf{Liquid:} incompressible
\[
  \rho = \rho(T)
.\] 

(no pressure dependence!)


Now we have $3$ equations relating $4$ variables ($ \rho, p, \vec{u}, T$ ).
Still need one more equation.

\section{Total Energy Equation (Thermal Energy Equation).}%

Last equation derives from thermodynamics

\[
\diff[]{u}{t} = \underbrace{Q}_\text{heat input / unit time}
+ \underbrace{W}_\text{work done / unit time}
.\] 

where $u$ is total internal energy of a parcel of fluid or container. Sometimes
written as

\[
du = -p dV + TdS
.\] 

In fluid dynamics, when applied to parcels,

\[
  \rho \frac{D \rho}{Dt} = \underbrace{-p\nabla\cdot \vec{u}
  + \underbrace{Q}_\text{heat}}_\text{external}
  + \underbrace{\underbrace{\phi}_\text{viscous heating} -\nabla\cdot
  \vec{q}}_\text{internal}
.\]

$\phi$ comes from nearby parcels ...

\textbf{Special cases} 

for a perfect gas: $e= \underbrace{q}_\text{specif heat at constant volume}T$ 

\[
\rho q \frac{DT}{Dt} = -\nabla
.\] 

hooks law $q = -k\nabla T$ 

\[
\implies
.\] 

finally $\phi:$ is usually negligible, and if not, can be written in terms of
$ \rho, \vec{u}$ 

$\implies$ energy eq relates $ \rho, T, \vec{u}, p$ 

\section{Conservation Laws}%

Only if we ignore dissipation (viscosity?)
\subsection{Conservation of Mass}%
\subsection{Conservation of Momentum in Conservative Form}%
\begin{align*}
  \rho \frac{D\vec{u}}{Dt} &= - \nabla p \diffp[]{ \rho \vec{u}}{t} + \nabla\cdot ( \rho \vec{u} \vec{u} + pI)
\end{align*}

$a = \rho \vec{u}$ (a vector), and $F_a = \rho \vec{u} \vec{u} + pI$ (a tensor)

(Check what this looks like in Cartesian coordinates)

\subsection{Conservation of Energy in Conservative Form}%

(ignore viscosity for now although it can be included)

\begin{align*}
  \rho \frac{De}{Dt} &= -p\nabla\cdot \vec{u} \\
  \rho (\diffp[]{e}{t}+ \mu\cdot\nabla e) &= -p\nabla\cdot \vec{u} \\
  \diffp[]{ \rho e}{t}+\nabla\cdot ( \rho \vec{u} e) &= -p\nabla\cdot \vec{u} \\
\end{align*}

