% !TEX root = /home/computer/ucsc/master-2/quarter-2/advanced-fluids/master.tex
\lecture{4}{Thu 03 Feb 2022 13:35}{Chapter 5 - part 2 (nonlinear convection}

\subsectionfont{\fontsize{10}{10}\selectfont}


\subsection{Weakly nonlinear theory of RBC above onset (above $Ra_c$)}%
\label{sub:weakly_nonlinear_theory_of_rbc_above_onset}

We know from data that convective rolls are *steady* (settle in steady state
with finite amplitude).

\begin{itemize}
  \item Can we model this?
\end{itemize}


\subsection{Preliminaries: tools needed}%
\label{sub:preliminaries}

\begin{itemize}
  \item Solvability condition (Fredholm's alternative (here for ODE's, but
    generalizes))
\end{itemize}

Consider a linear ODE
\[
  L [u(x)] = F(x)
.\] 

defined on $ (a,b)$ with boundary conditions

\begin{align*}
  u(a) = 0 \\
  u(b) = 0
\end{align*}

\textbf{Theorem:} If $L$ is self-adjoint with respect to the inner-product
$ \langle .,. \rangle$ and if there is a non-zero solution to
\[
  L(u_{n}) = 0
.\]

the equation $L(u) = F$ only has solutions if
\[
\langle u_{n}, F \rangle = 0
.\] 

\textbf{Definition} an operator $L$ is self-adjoint w.r.t the inner product
$ \left .,.  \right .,.$, if 
\[
\langle Lv,u \rangle = \langle v, Lu \rangle
.\] 

where
\[
  \langle u, v \rangle = \int_{a}^{b} {g(x)v(x)u(x)} \: d{x} 
.\] 

\textbf{Note:} Self adjoint operator for ODEs are unitarily diagonalizable

\textbf{Equivalently} we know that eigenfunctions of $L$ correspoding to
different eigenvalues are orthogonal w.r.t to $ \langle., .  \rangle$.
Therefore there exists a basis of eigenfunctions for any function on $ (a,b)$
can be written as
\[
  f(x) = \sum_{n}a_{n}v_{n}(x)
.\] 

these are called generalized Fourier Series

\begin{proof}
  We know that the solution to $L [u (x)] = F(x)$, if it exists can be written
  as
  \[
  u (x) = \sum a_{n}v_{n}
  .\] 

  where $L [v_{n}(x)] = \lambda _{n}v_{n}(x)$ 

  We also know 
  \[
    F (x) = \sum_{n} b_{n}v_{n}(x)
  .\] 
  \[
    \implies L [ \sum_{n} a_{n}v_{n}(s) = \sum b_{n}v_{n}(x)]
  .\] 

  Take dot product with $v_{m} (x)$ 
  \[
  \sum_{n} a_{n} \lambda _{n} \langle v_{m}, v_{n} \rangle = \sum_{n} b_{n}
  \langle v_{m}, v_{n} \rangle
  .\] 

  \[
    \implies  a_{m} = \frac{b_{m}}{\lambda _{m}}
  .\] 

  which is fine unless $\lambda _{m} - 0$ But if there is a solution to the
  problem 
  \[
  Lu_{n} = 0
  .\] 
  then that means $u_{h}$ is an eigenfunction of $L$ with eigenvalue $0$. If
  thats the case then $a_{m}$ corresponds to that eigenvalue is undefined unless
  $b_{m} = 0$ as well. That happens when
  \begin{align*}
    F(x) &= \sum b_{n}v_{n} \\
         &=
  \end{align*}
\end{proof}

\begin{itemize}
  \item "Baby step" weakly nonlinear theory on simple PDE
\end{itemize}

Consider 
\[
\begin{cases}
  \diffp[]{u}{t} -\sin u = \frac{1}{R} \diffp[2]{u}{z} \\
  u(0) = 0, & u( \pi ) = 0
\end{cases}
.\] 

Steady state solution: $u= 0$. Assume $u$ is small and linearize around steady
state. Use $\sin(u) = u$, then
\[
\diffp[]{u}{t} -u = \frac{1}{R} \diffp[2]{u}{z}
.\] 

Seeking solutions of the kind
\[
  u(z,t) = \hat{u }(z)e^{\lambda t}
.\] 

then 

\[
\diff[2]{\hat u}{z} = R (\lambda -1) \hat u
.\] 

Then look for solutions that satisfy boundary conditions $u(0) = 0$, $u(\pi
) = 0$, then
\[
\hat u = \begin{cases}
  \sin( \sqrt{R (1 -\lambda)} \\
  \cos( \sqrt{R (1 -\lambda)} \\
\end{cases}
.\] 

$u (0) = 0 \implies $ no cosine. $u (\pi ) = 0 implies$
\[
\lambda _{n} = 1 - \frac{n^2}{R}
.\] 

If $R < 1$, then all eigenvalues $<0$ and all perturbations decay

If $1<R<4$, then only mode that grows is $n = 1$.
\begin{itemize}
  \item we expect very simple behavior (single mode excited)
  \item we want to study the nonlinear saturation of that mode, for $R
    = 1 + \epsilon $.

    ($R$ is just a little bit above critical.)
\end{itemize}

for this $R$, 
  \begin{align*}
    \lambda _{1} = 1 - \frac{1}{R} = 1 - \frac{1}{1+ \epsilon } \\
    = 1 - (1-\epsilon  + \epsilon ^2 + \dots) = \epsilon + \text{H.O.T}
  \end{align*}

  \begin{itemize}
    \item the mode is growing exponentially at rate $\epsilon $ 
      \[
        u(z,t) \approx e^{\epsilon t}
      .\] 
  \end{itemize}

  So now the PDE is 
  \[
  \epsilon \diffp[]{u}{T} - \sin u  = \frac{1}{1 + \epsilon } \diffp[2]{u}{z}
  .\] 

  and we assume
  \[
    u(z,T) = \epsilon ^{\alpha } u_0 + \epsilon ^{2 \alpha } u_{1} + \dots
  .\] 

  this assumes the nonlinear solutions has small amplitude of $ \epsilon $ is
  small. How small $u$
