% !TEX root = /home/computer/ucsc/master-2/quarter-2/advanced-fluids/master.tex
\lecture{2}{Thu 06 Jan 2022 13:40}{Non-dispersive Waves}

\subsectionfont{\fontsize{10}{10}\selectfont}

\section{Non-dispersive Waves}%

Satisfy dispersion relation.

\[
\omega = \alpha k
.\] 

where $ \omega$ is frequency and $k$ is the wave number. Examples include sound
waves, and electromagnetic waves.

\section{Sound waves in a homogeneous, invariant medium}%

homogeneous: $u$ is constant in space
invariant: $u$ is constant in time

\subsection{The wave equation for small amplitude pertubations} 

\textbf{Mass conservation} 
time change in density is equal to divergence of mass (negative)

\textbf{Momentum (ignoring gravity / viscosity)} 
density times substantial time derivative of velocity equals gradient of
pressure (negative)

consider the background has

\[
\begin{cases}
  p = p_{0} & \textbf{constant} \\
  \rho = \rho_{0} & \textbf{constant} \\
  \vec{u} = 0 & \textbf{no back flow} 
\end{cases}
.\] 

then let

\begin{align*}
  p(x,t) &= p_0 + \hat p(x,t) \\
  \rho(x,t) &= \rho_0+ \hat \rho(x,t) \\
  \vec{u} &= 0 + \hat \vec{u}(x,t)
\end{align*}

where $\hat p << p_0$, $\hat \rho << \rho_0$ (i.e. small perturbations) and $x$ is a vector

$\implies$ applying mass cons. and ignoring quadratic terms due to small
perturbations 

\[
  \diffp[]{\hat \rho}{t} = - \rho_0 \nabla \vec{\hat u}
.\] 

$\implies$ applying momentum conservation and ignoring quadratic terms due to
small perturbations 

\[
  \rho_0 \diffp[]{\hat u}{t} = - \nabla\hat p
.\] 


taking $\diffp[]{}{t}$ of mass conservation, then we get

\[
\diffp[2]{\hat \rho}{t} = \nabla^{2}\hat p
.\] 

which almost looks like the (hyperbolic) wave equation. We need EOS to relate
$\hat \rho$ and $\hat p$


Assume a perfect gas  $p=R \rho T$, then liberalizing this (applying small
perturbations)

\[
  p_0 + \hat p = R ( \rho_0 + \hat \rho)(T_0+\hat T)
.\]

in the background $p_0=R \rho_0 T_0$, removing higher order terms, we get

\[
  \hat p = R(\hat \rho T_0 + \rho_0 \hat T)
.\] 

Make assumptions to simplify the problem. Assume $\hat T = 0$ (at least very
close to zero). This would give us the desired relationship between $\hat \rho$
and $\hat p$. Assuming temperature fluctuations decay very rapidly through
radiation or diffusion). This is the isothermal assumption. So we have

 \[
\hat p = R \hat \rho T_0
.\] 

$\implies$ wave equation is

\[
\diffp[2]{\hat \rho}{t} = R T_0 \nabla^{2}\hat \rho
.\] 

where the wave speed (isothermal sound speed)

\[
  c_{T} = \sqrt{RT_0}
.\] 

for air $c_{T} \approx 290 \frac{m}{s}$ (approximately). Although air is not actually
isothermal.


Solve this using d'Alembert's technique (only works in $1$D). The idea is to
use a change of variable $ \eta = x- ct$ and $\zeta = x+ct$. Then we get

\[
  \frac{\partial^{2}p}{\partial\eta\partial\zeta} = 0
.\] 

integrating twice, we get

\begin{align*}
  p(\eta,\zeta) &= F(\zeta) + G(\eta) \\
  p(x,t) &= F(x+ct) + G(x-ct)
\end{align*}

True for any function satisfying $1$D wave equation over infinite domain $x\in
(-\infty,\infty)$. Suppose initial conditions are

\[
\begin{cases}
  p(x,0) = p_0(x) \\
  \diffp[]{p}{t}(x,0) = q_0(x)
\end{cases}
.\] 

applying these, we get

\[
\begin{cases}
  p_0(x) = F(x) + G(x) \\
  q_0(x) = cF'(x) -cG'(x)
\end{cases}
.\] 

integrating $q_0$ 

\[
\frac{1}{c}  \int_{0}^{x} {q_0(s)} \: d{s} = (F(x)-F(0)) - (G(x) -G(0))
.\] 

this yields

\[
  p(x,t) = \frac{1}{2}p_0(x+ct) + \frac{1}{2}p_0(x-ct) + \frac{1}{2c}
  \int_{x-ct}^{x+ct} {q_0(s)} \: d{s} 
.\]

\section{Monochromatic wave solution of the wave equation in an infinite domain}%

In electromagnetic waves, single frequency of light. For sound, analogously,
this is just a single sound frequency. Assume the ansatz

\[
  p(x,t) = \hat p e^{ikx-i \omega t}
.\] 

where $\hat p$ can be complex and is constant. At the end, just take the real
part of the solution $Re(p(x,t))$. Plugging this into the wave equation, we get

\[
\omega^{2} = c^{2}k^{2} \implies \omega = \pm ck
.\] 

This is the dispersion relation for monochromatic sound waves, and shows that
the waves are non-dispersive. By convention, choose $ \omega>0$, so

\[
\omega = c|k|
.\] 

with this choice, the sign of $k$ tells us the direction of propagation. Taking
the real part of $p(x,t)$ as a linear combination of $k>0$ (propagates to the
right) and $k<0$ (soln propagates to the left) then
consider $\hat p_{+}$ and $\hat p_{-}$ are either real, or pure imaginary.


\textbf{Define Phase Speed}

\begin{align*}
  p(x,t) &= \hat p e^{ikx-i \omega t} \\
         &= \hat p e^{i \theta(x,t)}
\end{align*}

where $\theta(x,t) = kx- \omega t$ is the \textbf{phase} of the wave. If
$\theta$ is constant then $p$ is constant, though if $\theta$ is constant, then
so is $kx- \omega t = \text{const}$.


Drawing lines for $k>0$, and $k<0$ in the $x,t$ plane for

\begin{align*}
  t &= \frac{kx -\text{const}}{ \omega} \\
    &= \text{sign}(k) \frac{x}{c} + \text{const}
\end{align*}


The constant phase propagates at velocity c, which propagates at  So .. phase
speed and group speed are the same. unique to non-dispersive waves.
